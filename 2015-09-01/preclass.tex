\documentclass{scrartcl}
\usepackage{dominatrix}
\begin{document}
  \begin{framed}
  \large
  CS 5220 Applications of Parallel Programming \hfill Fall 2015 \\
  Kenneth Lim (\href{mailto:kl545@cornell.edu}{kl545}) \hfill Computer Architecture Basics \hspace{-3ex}
  \end{framed}
  \begin{enumerate}
    \item See previous submission.
    \item See previous submission.
    \item There is a $p-1$ speedup factor.
    \item The minimum serial time is 2.75h. The minimum parallel time is 2.25h.
    \item
      \begin{figure*}[ht!]
        \centering
        \begin{subfigure}[c]{.5\textwidth}
          \centering
          \includegraphics{local-timings-heat}
        \end{subfigure}%
        \begin{subfigure}[c]{.5\textwidth}
          \centering
          \includegraphics{local-timings-line}
        \end{subfigure}
      \end{figure*}
    \item
      \begin{figure*}[ht!]
        \centering
        \begin{subfigure}[c]{.5\textwidth}
          \centering
          \includegraphics{totient-timings-heat}
        \end{subfigure}%
        \begin{subfigure}[c]{.5\textwidth}
          \centering
          \includegraphics{totient-timings-line}
        \end{subfigure}
      \end{figure*}
    \item A caveat: when attempting to run \verb|centroid.c| verbatim from the course repo on the cluster, one found that it was consistently giving a timing of 0 for all three functions. The initial hypothesis was that the resolution of the default C timer was insufficient. Following up on Piazza, \verb|centroid.c| was modified to use the OpenMP timer, giving the following results:
    \begin{enumerate}[label=(\alph*)]
      \item 4.440892e-16 to 1.443290e-15
      \item 4.440892e-16 to 1.443290e-15
      \item 1.332268e-15 to 1.443290e-15
    \end{enumerate}
    also known as the unfortunate case where the numbers do not really make sense. Theoretically, Implementation B should be the slowest because it does not exploit memory locality or look-ahead, wheres Implementations A and C should be faster because the compiler can optimize for array access from contiguous blocks of memory.
  \end{enumerate}
\end{document}
