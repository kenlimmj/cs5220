\documentclass{scrartcl}
\usepackage{dominatrix,listings,xcolor,geometry,pdflscape,pbox,tikz}
\usepackage{pgfplots}

\pgfplotsset{
  every axis/.append style={font=\small},
  compat=newest
}

\lstset{%
  language=[LaTeX]TeX,
  keepspaces,
  basicstyle=\ttfamily,
  showspaces=false,
  showstringspaces=false,
  showtabs=false,
  tabsize=2,
  captionpos=b,
  breaklines=true,
  breakatwhitespace=false,
  commentstyle=\color{magenta},
  stringstyle=\color{cyan},
  showstringspaces=false,
  keywordstyle=[1]\color{blue},
  keywordstyle=[2]\color{magenta},
  keywordstyle=[3]\color{red},
  escapeinside={\@}{\@}
}

\begin{document}
  \begin{framed}
  \large
  CS 5220 Applications of Parallel Programming \hfill Fall 2015 \\
  Kenneth Lim (\href{mailto:kl545@cornell.edu}{kl545}) \hfill Performance Analysis Basics \hspace{-3ex}
  \end{framed}
  \begin{enumerate}
    \item Given $\alpha = 0.1$, we have:
    \[
      S(p) \leq \frac{1}{0.1 + 0.9/p},\quad 1<p<128
    \]
    % This result is shown in \autoref{fig:amdahl}, and the script used for graph generation is shown in \autoref{code:amdahl}.
    \begin{figure}[p]
      \centering
      \resizebox{0.75\textwidth}{!}{%
      \begin{tikzpicture}
        \begin{axis}[
          enlarge x limits=false,
          enlarge y limits=upper,
          ylabel=$S(p)$,
          xlabel=Cores,
          axis lines*=left,
          axis on top,
          tick align=outside
        ]
          \newcommand{\ALPHA}{0.1}
          \addplot[black,domain=1:128]
          {1/(\ALPHA + (1-\ALPHA)/x)};
        \end{axis}
      \end{tikzpicture}
    }
    \caption{Plot of idealized speedup vs. number of cores used for $p = 1$ to $p = 128$ and $\alpha = 0.1$. The graph asymtotes at $S(p) = 1/\alpha = 10$.\label{fig:amdahl}}
    \end{figure}
    \begin{figure}[p]
\begin{lstlisting}
  \begin{tikzpicture}
    \begin{axis}[
      enlarge x limits=false,enlarge y limits=upper,
      xlabel=Cores,ylabel=$S(p)$,tick align=outside,
      axis lines*=left,axis on top]
      \newcommand{\ALPHA}{0.1}
      \addplot[black,domain=1:128]
      {1/(\ALPHA + (1-\ALPHA)/x)};
    \end{axis}
  \end{tikzpicture}
\end{lstlisting}
    \caption{Graph Generation Code, using \LaTeX's TikZ and PGFPlots packages\label{code:amdahl}}
    \end{figure}
    \item Applying Gustafson's Law, we have:
    \[
      T = \frac{\alpha}{\alpha + \tau}
    \]
    Then the speedup is simply:
    \[
      S(p) = p - T(p - 1)
    \]
    \item Tuning should be avoided when:\begin{inparaenum}[(a)]
      \item the man hour cost of tuning exceeds the computation time saved;
      \item it potentially converts the codebase to spaghetti \emph{al dente};
      \item and when it does not target or significantly affect the code bottleneck.
    \end{inparaenum}
    \item The peak double precision FLOPS for an Intel Xeon Phi coprocessor 5110P is $1010.88\,\si{GF/\s}$\footnote{\url{https://www-ssl.intel.com/content/www/us/en/benchmarks/server/xeon-phi/xeon-phi-theoretical-maximums.html}}. where the performance is obtained by:
    \[
      16\,\frac{\si{FLOPS}}{\si{clock}} \times 60\,\si{cores} \times 1.053\,\si{GHz} = 1010.88\,\si{GF/\s}
    \]
    The peak double precision FLOPS for a single Intel E5-2620 v3 with Turbo Boost is $120\,\si{GF/\s}$\footnote{\url{http://download.intel.com/support/processors/xeon/sb/xeon_E5-2600.pdf}}. This came straight from Intel's data sheet without specifying how it is calculated, though it \emph{probably} uses only AVX2 instructions.

    Since we have a total of 96 cores and 15 accelerators, the overall peak FLOPS for the system is:
    \[
      96(120) + 15(1010.88) = 26683.2\,\si{GF/\s} = 26.683\,\si{TF/\s}
    \]
    \item My machine is a Late 2013 15" Apple Retina Macbook Pro. The processor is an Intel Core i7-4850HQ with a maximum clock speed of $3.5\,\si{G\Hz}$, 4 cores, 16 vector instructions per cycle, and a 64-bit wide AVX instruction set (which works out to 1 double-precision operand). This gives:
    \[
      3.5 \times 4 \times 16 \times 1 = 224\,\si{GF/\s}
    \]
    In addition, the machine also has an integrated GPU as well as a dedicated GPU, both of which we assume can serve as accelerators. The former is an Intel Iris Pro 5200 with peak FLOPS of $832\,\si{GF/\s}$\footnote{\url{http://www.anandtech.com/show/6993/intel-iris-pro-5200-graphics-review-core-i74950hq-tested/2}}, and the latter is an NVIDIA GeForce GT 750M, with peak FLOPS of $722.7\,\si{GF/\s}$\footnote{\url{https://www.techpowerup.com/gpudb/2224/geforce-gt-750m.html}}. Thus the total theoretical peak flop rate for my machine is:
    \[
      224 + 832 + 722.7 = 1778.7\,\si{GF/\s} = 1.78\,\si{TF/\s}
    \]
  \end{enumerate}
\end{document}
